\documentclass[
  % all of the below options are optional and can be left out
  % course name (default: 2IL50 Data Structures)
  course = {{MATH102 Calculus II}},
  % quartile (default: 3)
  quartile = {{2}},
  % assignment number/name (default: 1)
  assignment = 26,%Monday, %Wednesday,
  topic = {{11.10: Taylor and Maclaurin Series}},
  % student name (default: Some One)
  %name = {{Some One ; Other Person}},
  % student number, NOT S-number (default: 0123456)
  %studentnumber = {{0123456 ; 0314159}},
  % student email (default: s.one@student.tue.nl)
  %email = {{s.one@student.tue.nl ; o.person@student.tue.nl}},
  % first exercise number (default: 1)
  firstexercise = 1,
  term = 202
]{aga-homework}

\usepackage{graphicx}
\usepackage{advdate}
\usepackage{amsmath}
\usepackage[shortlabels]{enumitem}
\usepackage{afterpage}

\begin{document}
%\noindent {\hfill \vspace{-2\baselineskip} {\footnotesize\bf  Due \AdvanceDate[14]\today  \; midnight} \vspace{3mm}}

\begin{center}
  {\large
  \emph{Please indicate the members who are present. Also indicate the group coordinator.}
  \begin{tabular}{|l|l|}
    \hline
    % after \\: \hline or \cline{col1-col2} \cline{col3-col4} ...
    Group Number: & \hspace{3in} \\
     &  \\ \hline
    Members: &  \\
        &  \\ \cline{2-2}
        &  \\
        &  \\ \cline{2-2}
        &  \\
        &  \\ \cline{2-2}
        &  \\
        &  \\ \cline{2-2}
        &  \\
        &  \\ \cline{2-2}
        &  \\
        &  \\ \cline{2-2}
        &  \\
        &  \\ \cline{2-2}
        &  \\
        &  \\ \cline{2-2}
        &  \\
        &  \\ \cline{2-2}
        &  \\
        &  \\ \cline{2-2}
        &  \\
        &  \\ \cline{2-2}

    \hline
  \end{tabular}
  }
\end{center}

\fbox{
\begin{minipage}{\textwidth}

\mbox{}
\vspace{10mm}

Memorize Maclaurin Series listed in the table in section 11.10 (page 768).

\vspace{10mm}

%\vspace{2mm}
%\hrule
%\vspace{2mm}
%A series $\displaystyle \sum a_n$ is called {\bf \emph{absolutely convergent}} if and only if the series of the absolute values $\displaystyle \sum |a_n|$ is convergent.
%\vspace{2mm}
%\hrule
%\vspace{2mm}
%A series $\displaystyle \sum a_n$ is called {\bf \emph{conditionally convergent}} if $\displaystyle \sum a_n$ converges but $\displaystyle \sum |a_n|$ diverges.

\end{minipage}
}

\newpage

\problem Find the Maclaurin series for (Assume that $f$ has a power series expansion) $$\displaystyle f(x)=e^{-2x}$$

\newpage

\problem Find the Maclaurin series for (Assume that $f$ has a power series expansion) $$\displaystyle f(x)=2^{x}$$

\newpage

\problem Find the Maclaurin series for (Assume that $f$ has a power series expansion) $$\displaystyle f(x)=\sin 3x $$

\newpage

\problem Find the Taylor series for $f(x)$ centered at the given value of $a$.
$$f(x)=\ln x, \quad a=2$$

\newpage

\problem Find the Taylor series for $f(x)$ centered at the given value of $a$.
$$f(x)=e^{2x}, \quad a=2$$

\newpage

\problem Use the binomial series to expand the given function as a power series
$$ f(x)=\sqrt[3]{8+x}$$

\newpage

\problem Find the sum of the series
$$\sum_{n=0}^{\infty} \frac{(-1)^n\pi^{2n}}{6^{2n}(2n)!}$$
\newpage
\afterpage{\null\newpage}

\afterpage{\null\newpage}

\afterpage{\null\newpage}

\end{document} 