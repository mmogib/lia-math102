\documentclass[
  % all of the below options are optional and can be left out
  % course name (default: 2IL50 Data Structures)
  course = {{MATH102 Calculus II}},
  % quartile (default: 3)
  quartile = {{2}},
  % assignment number/name (default: 1)
  assignment = 16,%Monday, %Wednesday,
  topic = {{E1: Revision}},
  % student name (default: Some One)
  %name = {{Some One ; Other Person}},
  % student number, NOT S-number (default: 0123456)
  %studentnumber = {{0123456 ; 0314159}},
  % student email (default: s.one@student.tue.nl)
  %email = {{s.one@student.tue.nl ; o.person@student.tue.nl}},
  % first exercise number (default: 1)
  firstexercise = 1,
  term = 202
]{aga-homework}

\usepackage{graphicx}
\usepackage{advdate}
\usepackage{amsmath}
\usepackage[shortlabels]{enumitem}
\usepackage{afterpage}
\usepackage{multicol}
\usepackage{color}


\setlength{\columnseprule}{1pt}
\def\columnseprulecolor{\color{blue}}

\begin{document}
%\noindent {\hfill \vspace{-2\baselineskip} {\footnotesize\bf  Due \AdvanceDate[14]\today  \; midnight} \vspace{3mm}}

\begin{center}
  {\large
  \emph{Please indicate the members who are present. Also indicate the group coordinator.}
  \begin{tabular}{|l|l|}
    \hline
    % after \\: \hline or \cline{col1-col2} \cline{col3-col4} ...
    Group Number: & \hspace{3in} \\
     &  \\ \hline
    Members: &  \\
        &  \\ \cline{2-2}
        &  \\
        &  \\ \cline{2-2}
        &  \\
        &  \\ \cline{2-2}
        &  \\
        &  \\ \cline{2-2}
        &  \\
        &  \\ \cline{2-2}
        &  \\
        &  \\ \cline{2-2}
        &  \\
        &  \\ \cline{2-2}
        &  \\
        &  \\ \cline{2-2}
        &  \\
        &  \\ \cline{2-2}
        &  \\
        &  \\ \cline{2-2}
        &  \\
        &  \\ \cline{2-2}
        &  \\
        &  \\ \cline{2-2}
        &  \\
        &  \\ \cline{2-2}
        &  \\
        &  \\ \cline{2-2}
        &  \\
        &  \\ \cline{2-2}
        &  \\
        &  \\
    \hline
  \end{tabular}
  }
\end{center}

\newpage
\begin{multicols}{2}
\problem If $\displaystyle \int_{12}^{0}f(x)dx=-36$ and $\displaystyle \int_{0}^{9}f(x)dx=20$. Find $\displaystyle \int_{3}^{4}f(3x)dx$

\vfill
\mbox{}
\columnbreak

\problem Evaluate $\displaystyle \int_0^1 (x^{10}+10^x)dx$. 

\newpage
\problem Find the limit, if exists, $$\displaystyle \lim_{n\to \infty}\sum_{i=1}^{n}\frac{2}{n}\left(1+\frac{31i}{n}\right)^{-4/5}$$.

\vfill
\mbox{}
\columnbreak
\problem Find the average value of $\displaystyle f(x)=\frac{x}{(x^2+1)^3}$ from $1$ to $3$.
\newpage

\problem The base of a solid $S$ is bounded by $x = y^3, y = 1$ and the y-axis. If parallel cross-sections perpendicular to y-axis are equilateral triangles. Find the volume of the solid.

\vfill
\mbox{}
\columnbreak
\problem Using the method of cylindrical shells, find the volume of the solid generated by revolving the region bounded by 
$\displaystyle y =\sqrt{x}, x = 0$ and $y = 1$ about the
line $y = 2$.
\newpage

\problem Evaluate $\displaystyle \int \frac{dx}{1+\cos x}$ \\(DO IT IN TWO DIFFERENT METHODS).

\vfill
\mbox{}
\columnbreak
\problem Find $\displaystyle \int \tan^3x\sec^5x dx$.

\newpage
\problem Evaluate $\displaystyle \int_0^{\pi/2} \cos(3x)\cos(2x) dx$.

\vfill
\mbox{}
\columnbreak
\problem Describe the volume of the solid generated
by rotating the region bounded by the curves $y=4x-x^2$ and $y=x$ about the $y-$axis by TWO DIFFERENT INTEGRALS (do not evaluate).
\newpage
\problem Evaluate $\displaystyle \int_0^{\pi/2} (2-\sin\theta)^2d\theta$.

\vfill
\mbox{}
\columnbreak
\problem Find $\displaystyle \int \frac{dx}{\sqrt{x^2+9}}$ (using trigonometric substitution).
\newpage
\problem Evaluate $\displaystyle \int_0^{\pi} (3x+2)\cos\left(\frac{x}{2}\right) dx$.


\vfill
\mbox{}
\columnbreak

\problem Evaluate $\displaystyle \int_{\ln(\pi/4)}^{\ln(\pi/2)} e^x\tan^{-1}\left(e^x\right) dx$.

\vfill
\newpage
\end{multicols}
\afterpage{\null\newpage}

\afterpage{\null\newpage}

\afterpage{\null\newpage}

\end{document} 